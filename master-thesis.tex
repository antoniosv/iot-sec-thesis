% Institute of Computer Science thesis template
% authors: Sven Laur, Liina Kamm
% last change Tõnu Tamme 09.05.2017
%--
% Compilation instructions:
% 1. Choose main language on line 55-56 (English or Estonian)
% 2. Compile 1-3 times to get refences right
% pdflatex bachelors-thesis-template
% bibtex bachelors-thesis-template
%--
% Please use references like this:
% <text> <non-breaking-space> <cite/ref-command> <punctuation>
% This is an example~\cite{example}.

\documentclass[12pt]{article}

% A package for setting layout and margins for your thesis 
\usepackage[a4paper]{geometry}

%%=== A4 page setup ===
%\setlength{\paperwidth}{21.0cm} 
%\setlength{\paperheight}{29.7cm}
%\setlength{\textwidth}{16cm}
%\setlength{\textheight}{25cm}


% When you write in Estonian then you want to use text with right character set
% By default LaTeX does not know what to do with õäöu letters. You have to specify
% a correct input and font encoding. For that you have to Google the Web     
%
% For TexShop under MacOS X. The right lines are 
%\usepackage[applemac]{inputenc}
%\usepackage[T1]{fontenc} %Absolutely critical for *hyphenation* of words with non-ASCII letters.
%
% For Windows and Linux the right magic lines are   
% \usepackage[latin1]{inputenc}
% \usepackage[latin5]{inputenc}
%%\usepackage[utf8]{inputenc} %Package inputenc Error: Unicode char ´ (U+B4) not set up for use with LaTeX
\usepackage[utf8x]{inputenc}
\usepackage[T1]{fontenc} %Absolutely critical for *hyphenation* of words with non-ASCII letters.

% Typeset text in Times Roman instead of Computer Modern (EC)
\usepackage{times}

% Suggested packages:
\usepackage{microtype}  %towards typographic perfection...
\usepackage{inconsolata} %nicer font for code listings. (Use \ttfamily for lstinline bastype)


% Use package babel for English or Estonian 
% If you use Estonian make sure that Estonian hyphenation is installed 
% - hypen-estonian or eehyp packages
%
%===Choose the main language in thesis
\usepackage[estonian, english]{babel} %the thesis is in English 
%\usepackage[english, estonian]{babel} %the thesis is in Estonian


% Change Babel document elements 
\addto\captionsestonian{%
  \renewcommand{\refname}{Viidatud kirjandus}%
  \renewcommand{\appendixname}{Lisad}%
}


% If you have problems with Estonian keywords in the bibliography
%\usepackage{biblatex}
%\usepackage[backend=biber]{biblatex}
%\usepackage[style=alphabetic]{biblatex}
% plain --> \usepackage[style=numeric]{biblatex}
% abbrv --> \usepackage[style=numeric,firstinits=true]{biblatex}
% unsrt --> \usepackage[style=numeric,sorting=none]{biblatex}
% alpha --> \usepackage[style=alphabetic]{biblatex}
%\DefineBibliographyStrings{estonian}{and={ja}}
%\addbibresource{bachelor-thesis.bib}


% General packages for math in general, theorems and symbols 
% Read ftp://ftp.ams.org/ams/doc/amsmath/short-math-guide.pdf for further information
\usepackage{amsmath} 
\usepackage{amsthm}
\usepackage{amssymb}

% Optional calligraphic fonts    
% \usepackage[mathscr]{eucal}

% Print a dot instead of colon in table or figure captions
\usepackage[labelsep=period]{caption}

% Packages for building tables and tabulars 
\usepackage{array}
\usepackage{tabu}   % Wide lines in tables
\usepackage{xspace} % Non-eatable spaces in macros

% Including graphical images and setting the figure directory
\usepackage{graphicx}
\graphicspath{{fig/}}

% Packages for getting clickable links in PDF file
%\usepackage{hyperref}
\usepackage[hidelinks]{hyperref} %hide red (blue,green) boxes around links
\usepackage[all]{hypcap}


% Packages for defining colourful text together with some colours
\usepackage{color}
\usepackage{xcolor} 
\definecolor{dkgreen}{rgb}{0,0.6,0}
\definecolor{gray}{rgb}{0.5,0.5,0.5}
\definecolor{mauve}{rgb}{0.58,0,0.82}


% Standard package for drawing algorithms
% Since the thesis in article format we must define \chapter for
% the package algorithm2e (otherwise obscure errors occur) 
\let\chapter\section
\usepackage[ruled, vlined, linesnumbered]{algorithm2e}

% Fix a  set of keywords which you use inside algorithms
\SetKw{True}{true}
\SetKw{False}{false}
\SetKwData{typeInt}{Int}
\SetKwData{typeRat}{Rat}
\SetKwData{Defined}{Defined}
\SetKwFunction{parseStatement}{parseStatement}


% Nice todo notes
\usepackage{todonotes}

% comments and verbatim text (code)
\usepackage{verbatim}


% Proper way to create coloured code listings
\usepackage{listings}
\lstset{ 
  %language=python,                % the language of the code
  %language=C++,
  language=Java,
  basicstyle=\footnotesize,        % the size of the fonts that are used for the code
  %numbers=left,                   % where to put the line-numbers
  %numberstyle=\footnotesize,      % the size of the fonts that are used for the line-numbers
  numberstyle=\tiny\color{gray}, 
  stepnumber=1,                    % the step between two line-numbers. If it's 1, each line 
                                   % will be numbered
  numbersep=5pt,                   % how far the line-numbers are from the code
  backgroundcolor=\color{white},   % choose the background color. You must add \usepackage{color}
  showspaces=false,                % show spaces adding particular underscores
  showstringspaces=false,          % underline spaces within strings
  showtabs=false,                  % show tabs within strings adding particular underscores
  frame = lines,
  %frame=single,                   % adds a frame around the code
  rulecolor=\color{black},		   % if not set, the frame-color may be changed on line-breaks within 
                                   % not-black text (e.g. commens (green here))
  tabsize=2,                       % sets default tabsize to 2 spaces
  captionpos=b,                    % sets the caption-position to bottom
  breaklines=true,                 % sets automatic line breaking
  breakatwhitespace=false,         % sets if automatic breaks should only happen at whitespace
  %title=\lstname,                 % show the filename of files included with \lstinputlisting;
                                   % also try caption instead of title
  keywordstyle=\color{blue},       % keyword style
  commentstyle=\color{dkgreen},    % comment style
  stringstyle=\color{mauve},       % string literal style
  escapeinside={\%*}{*)},          % if you want to add a comment within your code
  morekeywords={*,game, fun}       % if you want to add more keywords to the set
}


% Obscure packages to write logic formulae and program semantics
% Unless you do a bachelor thesis on program semantics or static code analysis you do not need that
% http://logicmatters.net/resources/ndexamples/proofsty3.html <= writing type rules => use semantic::inference
% ftp://tug.ctan.org/tex-archive/macros/latex/contrib/semantic/semantic.pdf
\usepackage{proof}
\usepackage{semantic} 
\setlength{\inferLineSkip}{4pt}
\def\predicatebegin #1\predicateend{$\Gamma \vdash #1$}

% If you really want to draw figures in LaTeX use packages tikz or pstricks
% However, getting a corresponding illustrations is really painful  


% Define your favorite macros that you use inside the thesis 
% Name followed by non-removable space
\newcommand{\proveit}{ProveIt\xspace}

% Macros that make sure that the math mode is set
\newcommand{\typeF}[1] {\ensuremath{\mathsf{type_{#1}}}\xspace}
\newcommand{\opDiv}{\ensuremath{\backslash \mathsf{div}}\xspace} 

% Nice Todo box
\newcommand{\TODO}{\todo[inline]}

% A way to define theorems and lemmata
\newtheorem{theorem}{Theorem}



%%% BEGIN DOCUMENT
\begin{document}

%===BEGIN TITLE PAGE
\thispagestyle{empty}
\begin{center}

\iflanguage{english}{%
\large
UNIVERSITY OF TARTU\\%[2mm]
Institute of Computer Science\\
Security and Mobile Computing Curriculum\\%[2mm]
}{%
TARTU ÜLIKOOL\\
Arvutiteaduse instituut\\
Informaatika õppekava\\%[2mm]
}%\iflanguage

%\vspace*{\stretch{5}}
\vspace{25mm}

\Large Jes\'{u}s Antonio Soto Vel\'{a}zquez

\vspace{4mm}

\huge Security of the openHAB Smart Home \\
\large A contribution for the user authentication and authorization

%\vspace*{\stretch{7}}
\vspace{20mm}

\iflanguage{english}{%
\Large Master's Thesis (30 ECTS)
}{%
\Large Bakalaureusetöö (9 EAP)
}%\iflanguage

\end{center}

\vspace{2mm}

\begin{flushright}
 {
 \setlength{\extrarowheight}{5pt}
 \begin{tabular}{r l} 
  \sffamily \iflanguage{english}{Supervisor}{Juhendaja}: & \sffamily Satish Narayana Srirama, PhD \\
  \sffamily \iflanguage{english}{Supervisor}{Juhendaja}: & \sffamily Danilo Gligoroski, PhD
 \end{tabular}
 }
\end{flushright}

%\vspace*{\stretch{3}}
%\vspace{10mm}

\vfill
\centerline{Tartu 2018}

%===END TITLE PAGE

% If the thesis is printed on both sides of the page then 
% the second page must be must be empty. Comment this out
% if you print only to one side of the page comment this out
%\newpage
%\thispagestyle{empty}    
%\phantom{Text to fill the page}
% END OF EXTRA PAGE WITHOUT NUMBER


%===COMPULSORY INFO PAGE
\newpage

%=== Info in English
\newcommand\EngInfo{{%
    \selectlanguage{english}
\noindent\textbf{\large Security of the openHAB Smart Home \\ \normalsize A contribution for the user authentication and authorization}

\vspace*{3ex}

\noindent\textbf{Abstract:}

\noindent
Many interpreting program languages are dynamically typed, such as Visual Basic or Python. As a result, it is easy to write programs that crash due to mismatches of provided and expected data types.  One possible solution to this problem is automatic type derivation during compilation. In this work, we consider study how to detect type errors in the \textsc{Whitespace} language by using fourth order logic formulae as annotations. The main result of this thesis is a new triple-exponential type inference algorithm for the fourth order logic formulae. This is a significant advancement as the question whether there exists such an algorithm was an open question. 
All previous attempts to solve the problem lead lead to logical inconsistencies or required tedious user interaction in terms of interpretative dance. Although the resulting algorithm is slightly inefficient, it can be used to detect obscure programming bugs in the \textsc{Whitespace} language. The latter significantly improves productivity. Our practical experiments showed that productivity is comparable to average Java programmer.   
From a theoretical viewpoint, the result is only a small advancement in rigorous treatment of higher order logic formulae. The results obtained by us do not generalise to formulae with the fifth or higher order. 

\vspace*{1ex}

\noindent\textbf{Keywords:}\\
\TODO{List of keywords}
%Layout, formatting, template

\vspace*{1ex}

\noindent\textbf{CERCS:}\TODO{CERCS code and name:~\url{https://www.etis.ee/Portal/Classifiers/Details/d3717f7b-bec8-4cd9-8ea4-c89cd56ca46e}}

\vspace*{1ex}
}}%\newcommand\EngInfo


%=== Info in Estonian
\newcommand\EstInfo{{%
\selectlanguage{estonian}
\noindent\textbf{\large Tüübituletus neljandat järku loogikavalemitele}
\vspace*{1ex}

\noindent\textbf{Lühikokkuvõte:} 

%\noindent ...

\TODO{One or two sentences providing a basic introduction to the field, comprehensible to a scientist in
any discipline.}
\TODO{Two to three sentences of
more detailed background, comprehensible to scientists in related disciplines.}
\TODO{One sentence clearly stating the general problem being addressed by this particular
study.}
\TODO{One sentence summarising the main result (with the words ``here we show´´ or their equivalent).}
\TODO{Two or three sentences explaining what the main result reveals in direct comparison to what was thought to be the case previously, or how the main result adds to previous knowledge.}
\TODO{One or two sentences to put the results into a more general context.}
\TODO{Two or three sentences to provide a broader perspective, readily comprehensible to a scientist in any discipline, may be included in the first paragraph if the editor considers that the accessibility of the paper is significantly enhanced by their inclusion.}

\vspace*{1ex}

\noindent\textbf{Võtmesõnad:}\\
\TODO{List of keywords}
%Layout, formatting, template

\vspace*{1ex}

\noindent\textbf{CERCS:}\TODO{CERCS kood ja nimetus:~\url{https://www.etis.ee/Portal/Classifiers/Details/d3717f7b-bec8-4cd9-8ea4-c89cd56ca46e}}

\vspace*{1ex}
}}%\newcommand\EstInfo


%=== Determine the order of languages on Info page
\iflanguage{english}{\EngInfo}{\EstInfo}
\iflanguage{estonian}{\EngInfo}{\EstInfo}


\newpage
\tableofcontents


% Remember to remove this from the final thesis version
\newpage
\listoftodos[Unsolved issues]
% END OF TODO PAGE 


\newpage
\section{Introduction}

\TODO{(ref) references}
\TODO{Motivation on going from state of the art to OH2}
%% \TODO{What is it in simple terms (title)?}
%% \TODO{Problem Statement}
%% \TODO{Why should anyone care? (E.g. security)}
%% \TODO{What was my contribution? (E.g. authentication/authorization)} 
%% \TODO{What you are doing in each section (a sentence or two per section)}

%Tip: if it's hard for you to start writing, then try to split it to smaller parts, e.g. if the title is ``Type Inference for a Cryptographic Protocol Prover Tool'' then the ``What is it'' can be divided into ``what is type inference'', ``what is cryptographic protocol'' and ``what is the prover tool''. These three can also be split to smaller parts etc.

%---------------------------

openHAB is an automation software that brings together and operates \emph{Things} for the purpose of building a smart home environment. Smart home, a subset of the Internet of Things paradigm, is gaining popularity not only as a futuristic toy, but as a real intelligent environment that can be put in use today. \emph{Things}, which will be detailed later, form the basic unit in a smart home environment. Things capture data from the environment and transmit it to another point. Security is defined in ~\cite{whitman2011principles} as having ``protection against adversaries'', i.e., those would, intentionally or not, cause harm. In this context, security refers actually to \emph{information security}, defined in the same source as a layer of security that aims to protect the confidentiality, integrity, and availability of information resources that may be in storage, processing or transmision. Thus, this work introduces the notion of reviewing, evaluating, and possibly improving the existing (information) security present in the openHAB automation software for the smart home.

The security of the openHAB software, in terms of data protection and privacy preservation, is currently undefined, as there are no direct sources that address these topics. It is desirable to have an overview on what kind of security mechanisms are present and enforced in openHAB, as well as which vulnerabilities might have an impact on its use and future adoption. Morevoer, as an open-source project, there is no clear indication that the security is being actively looked into for this project, or if it is feasible to do so.

It is known that security breaches may have a significant economic impact on a firm, as described by \cite{GOEL}. Data loss or theft, tampering, and unauthorized operations are just some of the possible ocurrences led by the lack of proper security mechanisms in place. In the case of openHAB, a smart home application, it is not quite quantifiable how expensive it results to have a security breach occur at any level. The consequences may go from user discomfort to identity theft, or worse.

Applications for the Internet of Things are still very recent, and still not much is not known about the possibilities it will bring. This has led to ongoing efforts, such as openHAB, to focus mostly on the system functionalities, rather than the user experience, security, or many other non-functional requirements. Ideally, there should be a framework that can be used to evaluate the security of IoT applications. At the time, this has not been established due to the vast differences in the architectures and implementations. Indeed, as the environment grows more complex, so does the attack surface areas and possible vulnerabilities.

Not knowing the extent of how \emph{secure} openHAB or any other application of IoT is deters its adoption, and raises concerns about existing instances. Security by obscurity has never been the a reasonable attempt to protect information assets from adversaries. And indeed, as an open-source project, any party can freely view the code to try to find hidden vulnerabilities. For this reason, some effort could be spared for reviewing the security of existing models for Internet of Things, including concrete applications, such as openHAB.

The contribution of this work are mainly two: an overview of the state of the art sesecurity in Internet of Things, and the implementation of a token-based authentication mechanism for openHAB. As part of the second part of the contribution, a fine-grained access control scheme is proposed for authorization of resources of the smart home environment. 

This work is divided in ... sections. Section 2....


%---------------------------

\newpage
\section{State of the art} 


The Internet of Things, commonly referred as IoT, is a dynamic and heterogenous environment where \emph{sensing} devices may interact with each other for some particular purpose. The devices range from radio identification devices (RFID), infrared sensors, global positioning systems; to smart watches, smartphones, smart televisions, etc. The most important aspect for these devices is that they can gather some kind of data from the real world, and that are capable of transmitting it to another point. Usually, the devices may interact without human interventation, also known as Machine-to-Machine communication. Due to the wide diversity of capabilities among sensing devices, interoperability is a common challenge, especially due to vendor lock-in and obscure interfaces. Still, it is desirable that the devices can transparently communicate among themselves in a local scope, e.g. inside a wireless sensor network (WSN), and in some cases, even to an exteral scope, e.g. through the Internet. Typically, these devices do not directly communicate with each other, but do so through a gateway, which serves as a communication medium. Additionally, the gateway is also responsible for making the necessary translations between protocols if the devices need to send or receive data through the Internet.

The actual devices employed and the nature of the data gathered depend on the specific application for the IoT. These applications may be categorized into smart home, smart grid, smart city, smart transportation, and so on. Regardless of the application, data has to be gathered from the real world and transmitted to another point. Considering the diversity of devices and applications, it is no trivial task to unify or to support interoperability between devices, especially if they make use of different protocols for communication. Indeed, interoperability, device naming in a network, finding other devices, are just some examples of the difficulties that may be found when instantiating an IoT application. To address these issues, a variety of architectures and solutions have been proposed in (ref). 

Given this amount of issues present in the architecture and development of IoT applications, it is no surprise that much of the focus given both in research and industry has been towards the functional requirements. Non-functional requirements such as performance, user experience, security, among others, have been left as an afterthought.



\TODO{look for reference on law about privacy}
Due to recent developments on laws pertaining data security and privacy, more emphasis has given to the security of information systems. Thus, incorporating security into a software product is no longer a courtesy, but a duty toward the users. Data encryption, authentication, authorization mechanisms, non-repudiation, are just some of the possible measures that should be taken into account when designing a secure system, and the IoT is no exception to this. In this context, the privacy and confidentiality of the data gathered by the sensing devices come to mind. Depending on the nature of this data, the privacy might be essential to maintain, especially during transit to other devices or points in the network.

The smart home is one of the many applications for the IoT. A refrigerator that sends an message to your phone when it no longer has any milk. An air conditioner that turns on whenever the it learns that the outside temperature is rising above thirty Celsius degrees. A speaker that plays music whenever you start cooking. These are just some of the possible scenarios that may occur inside a home with \emph{sensing} devices that are capable of interacting with each other. This is no longer a vision for the future or a secluded experiment. There are already several existing solutions that attempt to bring together all these devices for bigger purpose. One such existing solution is the openHAB2 software stack.

OpenHAB2 (OH2) is such a product that focuses on interoperability among all kinds of devices from different vendors. It does accomplishes the interaction through logical modules called \emph{bindings}. For example, a smart television from Samsung may not be able to interact with other devices out of box, but it may be able to do so if the appropriate binding is developed for the openHAB2 environment. OpenHAB2, as it name implies, is an open-source software that serves as \emph{hub} that brings together a diverse range of devices through the use of bindings. A binding makes a link to a Thing that may be of either physical or logical nature. For instance, a light switch is undoubtedly a physical Thing, and its state of being turned on or off is the data it can make available. Consider however, a weather service from the Internet that provides weather information such as temperature, humidity, and precipitation probability. This provider of data is undoubtedly not of physical nature, but still fits properly into the model of a smart home. Thus, a binding may incorporate such a weather service as a Thing in the logical sense.

\TODO{Do I need reference here?}
Ironically, openHAB2 has been often been labeled as \emph{Intranet of Things}, precisely because it is intended to be used in a private network, without access from external sources that traverse through, e.g. the Internet. In short, openHAB2 cannot be accessed from the Internet, and the main reason for this is that there is no authorization mechanism in place for alllowing or forbidding access for users. Thus, any individual that can access the openHAB2 instance is capable of doing any changes and viewing any piece of data, without any lock in place. Therefore, the damage is controlled by making the instance accessible only from inside the private network. Thus, the security is as strong as the security of network. An intruder gaining access to the network implies in gaining access to all of the openHAB2 capabilities, breaching the privacy and confidentiality of the data in use. 

\TODO{check things needed to make an application open to the internet}


\TODO{IoT Applications: smarthome, etc}


\newpage
\section{State of the art} 
\TODO{Short description of what this section is about}

\TODO{Existing smart home solutions and comparison in securiy}

\subsection{Title of Subsection 1}

Some text...

\subsubsection{Title of Subsubsection 1}

Some text...

\subsubsection{Title of Subsubsection 2}

Some text...



\subsection{Title of Subsection 2} 

Rule: If you divide the text into subsections (or subsubsections) then there has to be at least two of them, otherwise do not create any. 

Tip: You can also use paragraphs, e.g.
\paragraph{Type rules for integers.} Some text ...

\paragraph{Type rules for rational numbers.} Some text here too...


\section{Architecture of Eclipse SmartHome and openHAB2}

\subsection{ESH}

\subsection{OH2}

\section{Methodology}
\TODO{Describe how we got to the point of implementing auth.}
As a first step towards analyzing the security of IoT models, architectures, and applications, a study on the state of the art was conducted. From it, various security mechanisms came to light, which served as the foundation toward data confidentiality, peer authentication, non-repudiation, etc. However, the studied proposals tended to oversimplify and deviate from the issues present in existing solutions. This observation made it clear that there was a disconnect from publications and from the products. Most products, however, do not make public the internal workings, and tend to make their own architectural decisions instead of following standards for, e.g. encoding data transmitted between things. Among the existing solutions for a smart environment, the openHAB smart home software was chosen as a case study in terms of security. The decision was made for various reasons: firstly, it is open-source, and thus it is possible to conduct white-box tests, secondly, it is vendor-agnostic, and finally, because of the active participation of the community in this project.

\subsection{Security of OH2}
\TODO{Analysis of bindings}
A binding is the logical piece of the system that links a \emph{thing} to openHAB. Through the User Interface or REST API calls, a user is able to view, and possibly modify the channels, i.e. values, of the things connected to the system. Temperature and humidity values, state of light switches, currently playing media, etc., are some examples of these channels. In the case of the least complex adversary, it may be assumed that it is possible to eavesdrop the incoming and outgoing data packets through the network. Thus, the first effort was to see how the data is moving around the system. Through the use of Wireshark and tcpdump, it was observed that the transit of data ocurred in two possible ways: through the cloud, or through the openHAB instance. Some devices, such as light switches, do not require to communicate with a server in the Internet to set or unset the state of the switch. As this may be done internally, the binding provides the means to operate the thing directly through the User Interface of openHAB. The other case involves devices which need to communicate to a remote server through the Internet to push its data. The binding, in this case, connects to the remote server through the use of proprietary API, and gets the data required from it. This is more evident in a logical thing, such as a weather service. The binding for the weather service connects to the server through an API to query data about temperature, humidity, etc., of a particular location.

Delegating the data to a remote server was the first obstacle in the analysis of the security in openHAB. Because of simplifications in the literature about IoT, it is typically assumed that for the architecture of IoT applications \TODO{REFERENCE} there is no connection to the Internet to accomplish a task that may be performed locally. However, due to different vendors and a variety of devices, the actual solution tends to depend on a remote connection. For this reason, the security frameworks \TODO{reference} in the literature do not quite fit in this scenario. This does not cause worse security implications, however; the analysis should be more flexible in that case.

Internal communication between things and the openHAB instance is typically done under a wireless network that encrypted with AES, for example. In that case, an eavesdropperis Ever is only able to get the transmitted data if it can break AES, which is computationally infeasible for a sensible amount of time for even a 128-bit key. Thus, data confidentiality in this scenario depends entirely on the security of the wireless network where the openHAB instance and the thing reside. Evidently, if Eve gains access to this private network, all intercepted communication is plainly visible to Eve.

The other scenario, in addition to eavesdropping of data traveling between thing and router, is also vulnerable during transmission of data through the Internet. Thus, an eavesdropper Eve that does not have access to the private network may still find a way to learn the data after it has left the router and into the Internet. Coming back to the example of the weather service, a binding may be programmed to get current temperature and humidity every 10 minutes. The request leaves the openHAB instance directly into the router, and then it travels through distinct points in the Internet. The remote server accepts the request if valid, and returns a response with the appropriate values in a format such as JSON or XML. If this request is not encrypted (e.g., through TLS), then eavesdropper Eve may easily learn the data sent back.

The last scenario hints at the implication of guaranteeing secure communication between things, the openHAB instance, and the remote servers. Indeed, if the request performed by a binding is pointed at a location through HTTPS, then the request will perform the TLS protocol, encrypting the communication. The main question in this case is then, is it guaranteed that the request will point to an HTTPS location? The answer to this could only be found by looking at the source code of the bindings present in openHAB. When analyzing the source code, it becomes evident that the URL chosen to direct the request is decided at the time the binding was written. This implies that the security of each binding is independent from each other. If a binding points to a plain HTTP URL, then it is only that binding that is subject to effective eavesdropping, and it would not affect other bindings added to the system.

\begin{figure} [htb]
\begin{lstlisting}
  public class Connection {
    private final String iCloudApiURL = "https://fmipmobile.icloud.com/fmipservice/device/";
    private final String iCloudAPIRequestDataCommand = "/initClient";
    private final Gson gson = new GsonBuilder().create();
    private final String dataRequest = gson.toJson(ICloudDataRequest.defaultInstance());
    
    private final byte[] authorization;
    private URL iCloudDataRequestURL;
    
    public Connection(String appleId, String password) throws MalformedURLException {
      iCloudDataRequestURL = new URL(iCloudApiURL + appleId + iCloudAPIRequestDataCommand);
    } 
    
    public String requestDeviceStatusJSON() throws IOException {
      HttpsURLConnection connection = connect(iCloudDataRequestURL);
      String response = postRequest(connection, dataRequest);
      connection.disconnect();    
      return response;
    }
  }
\end{lstlisting}
\caption{HTTP connection for iCloud binding.}
\label{lst:https_binding}
\end{figure}

Code snippet~\ref{lst:https_binding} is an example of a binding, in this case for iCloud. This binding is meant to establish a connection to the iCloud services, for example, to learn the status of a device. The method \texttt{requestDeviceStatus} is responsible for establishing the connection and returning the result as a JSON structure in a String variable. In the context of security, the important thing to note is that the connection is established through the use of the \texttt{HttpsURLConnection} class, which supports https-specific features, such as the encrypted communication through the TLS protocol. In this case, it is expected that the communication will be encrypted and an eavesdropper will not be able to read the plain data. Indeed, through the use Wireshark it was confirmed that the packets sent between the openHAB instance and the remote server were using the TLS protocol to communicate.

\TODO{REFERENCE}
As hinted, the use of HTTPS in bindings is of great importance due to the underlying Transport Layer Security protocol, also known plainly as TLS. According to the specification by the IETF, TLS provides communications security over the internet, and it is designed to prevent eavesdropping, tampering, or message forgery. The specifics of the protocol are of no importance in this work, thus it suffices to stress the fact that relying on it will guarantee the confidentiality and integrity of the data sent between the openHAB instance and remote servers.

\subsection{openHAB: Intranet of Things}

\TODO{REFERENCES}
An openHAB instance is typically installed on a small server, and could even be installed on a Raspberry Pi, deployed on some port, 8080 by default. Due to the configuration of openHAB, this port may only be accessed by end devices in the same wireless network. It has been asked countless times (REFERENCE) if it is possible to access the openHAB instance from the outside, that is: through the Internet. Exposing an application to the outside may be trivial from a functional standpoint, but it carries its own set of security risks. Denial-of-Service attacks, unrestricted URL access, injection, session hijacking, etc., are only some of the possibilities that could affect an application open to the Internet. These risks are well documented by projects such as the OWASP Top Ten (REFERENCE). In the case of openHAB particularly, an adversary does not need to explore too much before finding out an apparent vulnerability: the lack of authentication, and therefore, absence of access control.

According to the openHAB documentation, secure remote access is a problem that has been considered for a long time, and thus, some solutions are available (REFERENCE). These solutions are mostly three: VPN connection, myopenHAB Cloud Service, and running openHAB behind a reverse proxy. Without going into specifics, the idea in all of them has something in common: to make the transmission channel as safe as possible to prevent any unauthorized party from making use of it. The logic behind it is very simple: since there is no authorization mechanism in place for openHAB, then any party that can access the instance has control over all its features. Thus, the security against adversaries is as strong as the the security of the channel is.

Relying purely on the channel makes it impossible to make the openHAB instance \emph{open} to the Internet, as it is considered to  be a public and insecure medium. Note, however, that enforcing some kind of access control policy does not suffice to deem it safe against adversaries to open the instance to the Internet. The expected use for openHAB at the moment, however, takes the appearance of ``Intranet of Things'', as it cannot be accessed from the outside. A more interesting scenario would be to have two instances of openHAB communicating with each other to complement their functions with the data gathered separately.

At a first glance, one would think that authentication and access control should be implemented and be responsibility of the end product, i.e. openHAB. It turns out, however, that as a core feature that involves restricting access to REST end points and servlet extensions, it is more apropriate to fit the authentication and authorization logic inside the Eclipse SmartHome core. As previously stated in section (REFERENCE), the Eclipse SmartHome is a subset of the openHAB distribution that holds the core functionalities of automation of sensing devices. Thus, access control, and inherently, authentication, became to be a focus in the Eclipse SmartHome community. 

\subsection{Community Discussion on Role-Based Access Control}
\TODO{https://github.com/eclipse/smarthome/issues/579}
Starting from the situation that there is no access control mechanism in place, the community has long discussed on the implications of implementing authentication, of any kind, and role-based access control. As the project has greatly advanced without any foresight on access control, it has become increasingly difficult to directly implement any simple solution directly. In fact, not much documentation and examples are offered for authentication and access control for OSGi-based projects.

One such project is Apache Karaf, a container for the OSGi runtime, provides security based on JAAS (Java Authentication and Authorization Service). This embdedded security system can internally control access to OSGi services, console commands, etc. This is an interesting scenario as it relies on the basic authentication framework offered by Java, instead of relying in more complete products like Apache Shiro or Spring Security.

The community in openHAB was inspired by JAAS-based attempts at security and proposed a solution that made use of annotations and \emph{Basic} authentication. The changes of several OSGi bundles were made a pull request in eventually merged into the master branch of the project (REFERENCE). First of all, the changes themselves were designed as a sort of \emph{authentication API}, rather than a unique, concrete implementation. Meaning that a good portion of the code were made up of interfaces and abstract classes that defined methods to create and manage credentials, as well as doing authentication. One such concrete implementation offered with these changes is based on the JAAS realm with \emph{Basic} authentication. Basic authentication, in this case, means that the credentials are enclosed inside an HTTP request as a pair of the form \texttt{username:password}. These credentials are enclosed as part of the HTTP request header, and the concrete implementation is meant to extract these details from header to instantiate a \texttt{Credentials} object. Moreover, by relying on the JAAS realm, it was made possible to use \emph{annotations} in the code. These annotations serve to regulate access depending on the roles that the authenticated user has. If the authenticated user has the required role, then access is granted to the method or resource. The code snippet~\ref{jaas_roles} shows that to add a new Thing to the openHAB environment, the role of \emph{admin} is needed. If the user is not authenticated, or has a different role, then access to the method is forbidden. 

\begin{figure} [htb]
\begin{lstlisting}
    @POST
    @RolesAllowed({ Role.ADMIN })
    @Consumes(MediaType.APPLICATION_JSON)
    public Response create(String language, ThingDTO thingBean) {
      // Thing is added here
    }
\end{lstlisting}
\caption{HTTP connection for iCloud binding.}
\label{lst:jaas_roles}
\end{figure}

There are several problems with this approach, however. First, access control is not managed through a database or any other dynamic means, but is instead static. It is defined inside the source code, and there is no way to change permissions at runtime. This means that the project would need to be built again in order to take in any changes to the authorization on any method that makes use of it. Secondly, it offers no clear view on how the fine-grained details would have its access controlled. For example, let it be of public access to obtain the status of a light bulb, but only authorized users may flip its switch from the user interface. Moreover, a different kind of Thing would prefer the opposite behavior: to hide its status, but make it public to control it. As openHAB is vendor and Thing-agnostic, relying on annotations makes it impossible to deal with such fine-grained details in authorization.

Following the design patterns in Apache Shiro, the authentication API for the Eclipse SmartHome was designed to have the means to plug in any authentication providers as desired. These providers may provide the service either locally or remotely (e.g. through OAuth). The goal was to have a flexible solution that may accept different kinds of authentication mechanisms to satisfy the many different use-case scenarios. The concrete implementations would be done by the products relying on the Eclipse SmartHome, such as openHAB. Different products may have different scenarios and constraints for authentication, and thus it makes sense to have some flexibility in this aspect. For example, a new user may prefer to login through his Google credentials instead of setting up a new account for the particular SmartHome instance. 

A common problem with the new authentication API was that there was no way to turn it off, and thus it automatically rejected all incoming requests without an authorization header (REFERENCE). Normally, this is not a problem for most web applications, as there typically is a redirect method that leads to a login page, where the user can set his credentials. The authentication API, however, offered no login form, as it only supported basic authentication out of the box. Indeed, without a way to inject the credentials into every HTTP request, the changes were mostly unusable. This was a problem especially for new users, who would have no idea on what to do whenever a ``Forbidden access'' page would come up. Originally, it was thought that if no authentication provider was available, then access control would not be enforced, making it an optional feature. However, it turned out that not detecting any authentication provider made no change whatsoever, leading to all requests to the REST end points being rejected. In the end, it was decided to disable the authentication API bundle from the default runtime, so it would not impede the normal functioning of openHAB.

From this experience, it was decided it would be desirable to have a way to \emph{turn off} access control completely, in the case that the user does not have the means to authenticate and manage permissions. At first glance, this is a very counter-intuitive feature to have, as any adversary could push toward disabling security, instead of having to break it through more advanced methods. 

\subsection{Misuse Cases}

\subsection{Architectural Implications of Authentication}

\subsection{Implementation of Authentication}

\subsection{Proposed Authorization Scheme}
\TODO{Mention what will be "secured": servlets, REST endpoints, to view and to modify, etc}
\section{Evaluation}
\newpage
\section{Extra}
\subsection{How to use references} \label{sec:using_ref}

\paragraph{Cross-references to figures, tables and other document elements.}
LaTeX  internally numbers all kind of objects that have sequence numbers:
\begin{itemize}
\item chapters, sections, subsections;
\item figures, tables, algorithms;
\item equations, equation arrays.
\end{itemize}
To reference them automatically, you have to generate a label using \texttt{$\backslash$label\{some-name\}} just after the object that has the number inside. Usually, labels of different objects are split into different namespaces by adding dedicated prefix, such as \texttt{sec:}, \texttt{fig:}. To use the corresponding reference, you must use command \texttt{$\backslash$ref} or \texttt{$\backslash$eqref}. For instance, we can reference this subsection by calling Section~\ref{sec:using_ref}. Note that there should be a nonbreakable space \texttt{\~} between the name of the object and the reference so that they would not appear on different lines (does not work in Estonian).          



\paragraph{Citations.}
Usually, you also want to reference articles, webpages, tools or programs or books. For that you should use citations and references. The system is similar to the cross-referencing system in LaTeX. For each reference you must assign a unique label. Again, there are many naming schemes for labels. However, as you have a short document anything works. To reference to a particular source you must use \texttt{$\backslash$cite\{label\}} or \texttt{$\backslash$cite[page]\{label\}}. 

References themselves can be part of a LaTeX source file. For that you need to define a bibliography section. However, this approach is really uncommon. It is much more easier to use BibTeX to synthesise the right reference form for you. For that you must use two commands in the LaTeX source
\begin{itemize}
\item $\backslash$bibliographystyle\{alpha\} or $\backslash$bibliographystyle\{plain\}
\item $\backslash$bibliography\{file-name\}
\end{itemize}
The first command determines whether the references are numbered by letter-number combinations or by cryptic numbers. It is more common to use \texttt{alpha} style. The second command determines the file containing the bibliographic entries. The file should end with \texttt{bib} extension. Each reference there is in specific form. The simplest way to avoid all technicalities is to use graphical frontend  Jabref (\url{http://jabref.sourceforge.net/}) to manage references. Another alternative is to use DBLP database of references and copy BibTeX entries directly form there.   
    
   
The following paragraph shows how references can be used. Game-based proving is a way to analyse security of a cryptographic protocol~\cite{GameB_1, GameB_2}. There are automatic provers, such as {CertiCrypt\-}~\cite{dummy} and ProVerif~\cite{proVerif}.



\newpage
\section{How to add figures and pictures to your thesis}


Here are a few examples of how to add figures or pictures to your thesis (see Figures~\ref{fig:fnCompModel}, \ref{fig:game-based_proofs}, \ref{fig:proveit_screenshot}).

Rule: All the figures, tables and extras in the thesis have to be referred to somewhere in the text.


\begin{figure} [ht] %try to place the figure here (next option top of the page) 
\begin{center}
\includegraphics[width=0.8\textwidth]{computational_model_function}
\caption{The title of the Figure.}
\label{fig:fnCompModel}
\end{center}
\end{figure}



\begin{figure} [!ht] %if [h] doesn't work, we can force with !
\begin{center}
\includegraphics[width=\textwidth]{game-based_proofs}
\caption{Refer if the figure is not yours~\cite{kamm12}.}
\label{fig:game-based_proofs}
\end{center}
\end{figure}


\begin{figure} [p]
\begin{center}
\includegraphics[width=\textwidth]{proveit_screenshot}
\caption{Screenshot of \proveit.}
\label{fig:proveit_screenshot}
\end{center}
\end{figure}

Tip: If you add a screenshot then labeling the parts might help make the text more understandable (panel C vs bottom left part), e.g.


\begin{figure} [htbp]
\begin{tabular}{c c}
%
\begin{minipage}{0.45\textwidth}
\includegraphics[width=\textwidth]{LCA_2_solutions}
\end{minipage}
%
&
\begin{minipage}{0.55\textwidth}
\centering
\begin{tabular}{ l | l |}
	Node & Decendants \\ \hline
  1 & 2, 3, 4 \\ \hline
  2 & 3, 4 \\ \hline
  3 & \\ \hline
  4 & \\ \hline
  5 & 3, 4, 6, 7 \\ \hline
  6 & 4 \\ \hline
  7 & 3 \\  \hline
  8 & 3, 4, 5, 6, 7\\ \hline
  9 & 3, 4, 5, 6, 7\\ \hline
\end{tabular}
\end{minipage}
\end{tabular}
%
\caption{Example how to put two figures parallel to each other.}
\label{fig:LCA_2_solutions}
\end{figure}


Example: A screenshot of \proveit can be seen on Figure~\ref{fig:proveit_screenshot}. The user first enters the pseudocode of the initial game in panel B. \proveit also keeps track of all the previous games showing the progress on a graph seen in panel A.

There are two figures side by side on Figure~\ref{fig:LCA_2_solutions}.



\clearpage %if newpage doesn't work
\section{Other Ways to Represent Data}

\subsection{Tables}

\begin{table}[h]
\centering
\caption{Statements in the \proveit language.}
\begin{tabular}{| l | l |}
	\hline
	\bf{Statement} & \bf{Typeset Example} \\
	\hline
	assignment & $a := 5 + b$ \\
	\hline
	uniform choice & $m <- M$ \\
	\hline
	function signature & $f : K \times M -> L$\\
	\hline
\end{tabular}
\label{tab:statements}
\end{table}


\subsection{Lists}

Numbered list example:
\begin{enumerate}
	\item item one; 
	\item item two;
	\item item three.
\end{enumerate} 

\subsection{Math mode}
Example:
\begin{equation}
a + b = c + d
\end{equation}
Aligning:
\begin{align*}
	a &= 5 \\
	b + c &= a \\
	a -2*3 &= 5/4
\end{align*}
Hint: Variables or equations in text are separated with \$ sign, e.g. $a$, $x - y$.

\paragraph{Inference Rules}
\[ 
	\inference[addition]{x : T & y : T}{x + y : T} 
\]
Bigger example:
\[
\inference[assign]{c := a + b & 
	\inference[addG]{a : \typeRat & 
		\inference[var]{b : \typeInt & \typeInt \subseteq \typeRat}{b : \typeRat}
		}{a + b : \typeRat}
	}{c : \typeRat}
\]


\subsection{algorithm2e}

\begin{algorithm} [!h]
	\caption{typeChecking} \label{alg:typeChecking}
	\KwIn{Abstract syntax tree}
	\KwResult{Type checking result; In addition, type table \typeF{type\_G} for global variables, \typeF{game} for the main game and \typeF{fun} for each $fun \in F$}
	\SetKwData{s}{s}
	\BlankLine
	
	\While{something changed in last cycle}{
		\lForEach{global statement \s} {
			\parseStatement{\s, \typeF{type\_G}}\;
		}
		\ForEach{function $fun$} {
		\lForEach{statement \s in $fun$} {
			\parseStatement{\s, \typeF{fun}}\;
		}
		}
		\lForEach{statement \s in game} {
			\parseStatement{\s, \typeF{game}}\;
		}
	}
	%\eIf{error messages were found}{\Return \False\;}{\Return \True\;}
\end{algorithm}

\subsection{Pseudocode}

\begin{figure} [htb]
\begin{lstlisting}
expression
  : NUMBER
  | VARIABLE
  | '+' expression
  | expression '+' expression
  | expression '*' expression
  | function_name '(' parameters ')'
  | '(' expression ')'
\end{lstlisting}
\caption{Grammar of arithmetic expressions.}
\label{fig:parser_exp}
\end{figure}

\subsection{Frame Around Information}

Tip: We can use minipage to create a frame around some important information.
\begin{figure} [h]
\frame{
\begin{minipage}{\textwidth}
\begin{enumerate}
	\item integer division ($\opDiv$) -- only usable between \typeInt types
	\item remainder ($\%$) -- only usable between \typeInt types
\end{enumerate}
\end{minipage}
}
\caption{Arithmetic operations in \proveit revisited.}
\label{fig:aritmOp_revisit}
\end{figure}



\clearpage
\section{Conclusion}

\TODO{what did you do?} 
\TODO{What are the results?}
\TODO{future work?}

\newpage

% BibTeX bibliography
\bibliographystyle{alpha} %plain=[1], alpha=[BGZ09]
\bibliography{master-thesis}

\addcontentsline{toc}{section}{\refname}


% Use Biblatex if you have problems with Estonian keywords
%\printbibliography %biblatex


% Use alternative local LaTeX bibliography
\begin{comment}
\begin{thebibliography}{9}
\bibitem{proVerif} 
  Bruno Blanchet. 
  Proverif: Cryptographic protocol verifier in the formal model.
  \url{http://www.proverif.ens.fr/}.
  (checked 15.05.2012)
\bibitem{GameB_1} GameB1
\bibitem{GameB_2} GameB2
\bibitem{dummy} dummy
\bibitem{kamm12} kamm12
\end{thebibliography}
\end{comment}


\newpage
%\appendix
%\section*{\appendixname}
\iflanguage{english}%
  {\section*{Appendix}
  \addcontentsline{toc}{section}{Appendix}
  }%
  {\section*{Lisad}
  \addcontentsline{toc}{section}{Lisad}}


\section*{I. Glossary}
\addcontentsline{toc}{subsection}{I. Glossary}

\newpage

%=== Licence in English
\newcommand\EngLicence{{%
\selectlanguage{english}
\section*{II. Licence}

\addcontentsline{toc}{subsection}{II. Licence}

\subsection*{Non-exclusive licence to reproduce thesis and make thesis public}

I, \textbf{Jes\'{u}s Antonio Soto Vel\'{a}zquez},

\begin{enumerate}
\item
herewith grant the University of Tartu a free permit (non-exclusive licence) to:
\begin{enumerate}
\item[1.1]
reproduce, for the purpose of preservation and making available to the public, including for addition to the DSpace digital archives until expiry of the term of validity of the copyright, and
\item[1.2]
make available to the public via the web environment of the University of Tartu, including via the DSpace digital archives until expiry of the term of validity of the copyright,
\end{enumerate}

of my thesis
\textbf{Security of the openHAB Smart Home}

supervised by Satish Narayana Srirama and Danilo Gligoroski

\item
I am aware of the fact that the author retains these rights.
\item
I certify that granting the non-exclusive licence does not infringe the intellectual property rights or rights arising from the Personal Data Protection Act. 
\end{enumerate}

\noindent
Tartu, 21.05.2018
}}%\newcommand\EngLicence


%=== Licence in Estonian
\newcommand\EstLicence{{%
\selectlanguage{estonian}
\section*{II. Litsents}

\addcontentsline{toc}{subsection}{II. Litsents}

\subsection*{Lihtlitsents lõputöö reprodutseerimiseks ja lõputöö üldsusele kättesaadavaks tegemiseks}

Mina, \textbf{Jes\'{u}s Antonio Soto Vel\'{a}zquez},

\begin{enumerate}
\item
annan Tartu Ülikoolile tasuta loa (lihtlitsentsi) enda loodud teose

\textbf{Tüübituletus neljandat järku loogikavalemitele}

mille juhendajad on Satish Narayana Srirama ja Danilo Gligoroski

\begin{enumerate}
\item[1.1]
reprodutseerimiseks säilitamise ja üldsusele kättesaadavaks tegemise eesmärgil, sealhulgas digitaalarhiivi DSpace-is lisamise eesmärgil kuni autoriõiguse kehtivuse tähtaja lõppemiseni;
\item[1.2]
üldsusele kättesaadavaks tegemiseks Tartu Ülikooli veebikeskkonna kaudu, sealhulgas digitaalarhiivi DSpace´i kaudu kuni autoriõiguse kehtivuse tähtaja lõppemiseni.
\end{enumerate}


\item
olen teadlik, et punktis 1 nimetatud õigused jäävad alles ka autorile.
\item
kinnitan, et lihtlitsentsi andmisega ei rikuta teiste isikute intellektuaalomandi ega isikuandmete kaitse seadusest tulenevaid õigusi. 
\end{enumerate}

\noindent
Tartus, 21.05.2018
}}%\newcommand\EstLicence


%===Choose the licence in active language
\iflanguage{english}{\EngLicence}{\EstLicence}


\end{document}

