%%%%%%%%%%%%%%%%%%%%%%%%%%%%%%%%%%%%%%%%%
% Beamer Presentation
% LaTeX Template
% Version 1.0 (10/11/12)
%
% This template has been downloaded from:
% http://www.LaTeXTemplates.com
%
% License:
% CC BY-NC-SA 3.0 (http://creativecommons.org/licenses/by-nc-sa/3.0/)
%
%%%%%%%%%%%%%%%%%%%%%%%%%%%%%%%%%%%%%%%%%

%----------------------------------------------------------------------------------------
%	PACKAGES AND THEMES
%----------------------------------------------------------------------------------------
\PassOptionsToPackage{subsection=false}{beamerouterthememiniframes}
\documentclass{beamer}
\usepackage{listings}
\usepackage{mdframed}

\renewcommand{\figurename}{Figure}
\renewcommand{\tablename}{Table}
\renewcommand{\lstlistingname}{Code Snippet} 

\def\titulo{{Securing openHAB Smart Home through User Authentication and Authorization}}
\def\autor{Jes\'{u}s Antonio Soto Vel\'{a}zquez}
\def\grado{Master of Science in Engineering (Computer Science)}
\def\matricula{B70630}
\def\fecha{June 2018}
\usepackage[numbers,sort]{natbib}
%\bibliographystyle{../estiloDeBibliografia}

\setbeamertemplate{caption}[numbered]


\definecolor{carolinablue}{rgb}{0.6, 0.73, 0.89}
\definecolor{lbcolor}{rgb}{0.9,0.9,0.9}
\lstset{
    tabsize=4,    
    language=Java,
    basicstyle=\scriptsize,
    upquote=true,
    %        aboveskip={1.5\baselineskip},
    columns=fixed,
    showstringspaces=false,
    extendedchars=false,
    breaklines=true,
    prebreak = \raisebox{0ex}[0ex][0ex]{\ensuremath{\hookleftarrow}},
    frame=single,
    numbers=left,
    numberstyle=\tiny,
    numbersep=5pt,
    showtabs=false,
    showspaces=false,
    showstringspaces=false,
    identifierstyle=\ttfamily,
    keywordstyle=\color[rgb]{0.0, 0.45, 0.73},
    commentstyle=\color[rgb]{0.09, 0.45, 0.27},
    stringstyle=\color[rgb]{0.627,0.126,0.941},
%    numberstyle=\color[rgb]{0.205, 0.142, 0.73},
    %        \lstdefinestyle{C++}{language=C++,style=numbers}’.
}


\mode<presentation> {
  \usetheme{Berlin}
  \usecolortheme{spruce}
%\useoutertheme[subsection=false]{miniframes}

%\setbeamertemplate{footline} % To remove the footer line in all slides uncomment this line
\setbeamertemplate{footline}[page number] % To replace the footer line in all slides with a simple slide count uncomment this line

%\setbeamertemplate{navigation symbols}{} % To remove the navigation symbols from the bottom of all slides uncomment this line
}

\usepackage{graphicx} % Allows including images
\graphicspath{{../fig/}}
\usepackage{booktabs} % Allows the use of \toprule, \midrule and \bottomrule in tables



%----------------------------------------------------------------------------------------
%	TITLE PAGE
%----------------------------------------------------------------------------------------

\title[]{\titulo} % The short title appears at the bottom of every slide, the full title is only on the title page

\author{\autor} % Your name
\institute[UT]{
  UNIVERSITY OF TARTU\\[3mm]
  Institute of Computer Science\\
  Security and Mobile Computing Curriculum\\[3mm]  
  \includegraphics[height=3cm]{UT_logo}
}

\date{\fecha} % Date, can be changed to a custom date

\begin{document}

\begin{frame}
\titlepage % Print the title page as the first slide
\end{frame}

\begin{frame}
\frametitle{Overview} % Table of contents slide, comment this block out to remove it
\tableofcontents % Throughout your presentation, if you choose to use \section{} and \subsection{} commands, these will automatically be printed on this slide as an overview of your presentation
\end{frame}

%----------------------------------------------------------------------------------------
%	PRESENTATION SLIDES
%----------------------------------------------------------------------------------------

%------------------------------------------------
\section{Introduction} % Sections can be created in order to organize your presentation into discrete blocks, all sections and subsections are automatically printed in the table of contents as an overview of the talk
%------------------------------------------------
\begin{frame}
\frametitle{Introduction: Internet of Things}
\textbf{Internet of Things.} Dynamic and heterogenous environment where \emph{sensing} devices interact with each other. 
\begin{itemize}
\setlength\itemsep{1.5em}
\item Devices: smartphones, smart watches, smoke alarm, security camera, heartbeat and temperature monitor, etc.
\item Applications: Smart home, smart transportation, smart healthcare, intelligent transportation, etc.
\item Challenges: interoperability, architecture, device naming, usability, security and privacy.  
\end{itemize}
\end{frame}
%------------------------------------------------
\begin{frame}
\frametitle{A Smart Home Application: OpenHAB}
\textbf{``a vendor and technology agnostic open source automation software for your home''}~\cite{oh_01}
\begin{itemize}
  \setlength\itemsep{1.5em}
\item May be deployed on very modest devices (e.g., Raspberry Pi)
\item Mostly works without an Internet connection.
\item Uses thing-specific \emph{bindings} to connect to devices.
\item Made up of individual projects, e.g.\ Eclipse SmartHome.
\item Written in Java under the OSGi architecture. 
\end{itemize}
\end{frame}
%------------------------------------------------
\begin{frame}
\frametitle{Problem Statement}
\begin{itemize}
  \setlength\itemsep{1.5em}
\item To review the existing security mechanisms of the openHAB smart home automation software.
\item To study and implement a JSON Web Token-based authenticator for Eclipse SmartHome, the core of openHAB, as a base for access control mechanisms.
\item To propose a usable authorization model to manage access permissions to things and resources in openHAB. 
\end{itemize}
\end{frame}
%------------------------------------------------
\begin{frame}
\frametitle{Motivation}
\begin{itemize}
  \setlength\itemsep{1.5em}
\item Developments of IoT applications focus mostly on functional requirements, leaving usability, performance, security, etc., for later.
\item Consequences in smart home range from mere user discomfort to identity theft, or worse.
\item Lack of security deters adoption of IoT applications, halts progress. 
\end{itemize}
\end{frame}
%------------------------------------------------
\begin{frame}
\frametitle{Contributions} % objectives?
\begin{enumerate}
  \setlength\itemsep{1.5em}
\item Analysis of the security mechanisms in use for the openHAB automation software.
\item Implementation of a client authenticator based on the JSON Web Token for Eclipse Smarthome.
\item Proposal of a fine-grained, yet usable authorization model for openHAB to manage usage permissions.
\end{enumerate}
\end{frame}
%------------------------------------------------
\section{Security Challenges in OpenHAB}
\begin{frame}
\frametitle{Security Challenges in OpenHAB}
\begin{itemize}
  \setlength\itemsep{1.5em}
\item Intranet of Things
\item Bindings
\item Access Control
\end{itemize}
\end{frame}
%------------------------------------------------
\begin{frame}
\frametitle{Intranet of Things}
\begin{itemize}
  \setlength\itemsep{1.5em}
\item Aims to contain private data locally.
\item Puts a limitation on use cases.
\item Assumes security of the private network.
\item Secure remote access: a challenge.
\end{itemize}
\end{frame}
%------------------------------------------------
\begin{frame}
\frametitle{Security of Bindings}
\textbf{Binding.} Logical modules to support inter-device interaction inside openHAB.
\begin{itemize}
  \setlength\itemsep{1.5em}
\item Each binding is defined for things that use a certain protocol, e.g.: KNX, Z-Wave, ZigBee, Panasonic TV, etc.
\item Some bindings use an API for remote connection to the vendor.
\item Remote communication may be done under HTTPS if specified by binding.
\end{itemize}
\end{frame}
%------------------------------------------------
\begin{frame}
\frametitle{Access Control}
\begin{itemize}
  \setlength\itemsep{1.5em}
\item Whoever gains access to private network gains access to smart home management.
\item Long-discussed topic in the openHAB community.
\item Not trivial: previous attempts have failed.
\item Preferably to be implemented inside the core framework: Eclipse SmartHome.
\item Main challenge: single \emph{authentication context} recognizable in all end points (servlets).
\item Community discussion led to a requirements document.
\end{itemize}
\end{frame}
%------------------------------------------------
\begin{frame}
\frametitle{Misuse Cases}

\begin{figure} [ht] 
\begin{center}
\includegraphics[width=0.75\textwidth]{esh_misuse_cases}
\caption{Misuse Cases for Access Control in ESH.}
\label{fig:misuse_cases}
\end{center}
\end{figure}

\end{frame}
%------------------------------------------------
\section{Proposed Security Mechanisms}
\begin{frame}
\frametitle{Proposed Security Mechanisms}
\begin{itemize}
  \setlength\itemsep{1.5em}
\item JSON Web Token-based Authenticator
\item RBAC-inspired Authorization Model
\end{itemize}
\end{frame}
%------------------------------------------------
\begin{frame}
\frametitle{Token-based Authentication Procedure}
\begin{figure} [ht] 
\begin{center}
\includegraphics[width=0.6\textwidth]{esh_auth_sequence}
\caption{Addition of authenticators into the architecture.}
\label{fig:esh_arch_authenticator}
\end{center}
\end{figure}
\end{frame}
%------------------------------------------------
\begin{frame}
\frametitle{Architectural Implications}
\begin{figure} [ht] 
\begin{center}
\includegraphics[width=0.8\textwidth]{esh_arch_authenticator}
\caption{Addition of authenticators into the architecture.}
\label{fig:esh_arch_authenticator}
\end{center}
\end{figure}
\end{frame}
%------------------------------------------------
\begin{frame}
\frametitle{Implementation of Authenticators}
\begin{itemize}
  \setlength\itemsep{1.5em}
\item Custom http context....
\end{itemize}
\end{frame}
%------------------------------------------------
\begin{frame}
\frametitle{Proposed Authorization Model}
\begin{itemize}
  \setlength\itemsep{1.5em}
\item alo polisia
\end{itemize}
\end{frame}
%------------------------------------------------
\begin{frame}
\frametitle{Evaluation}
\begin{itemize}
  \setlength\itemsep{1.5em}
\item alo polisia
\end{itemize}
\end{frame}
%------------------------------------------------
\section{Conclusion and Future Research Directions}
\begin{frame}
\frametitle{Contributions}
\begin{itemize}
  \setlength\itemsep{1.5em}
\item alo polisia
\end{itemize}
\end{frame}
%------------------------------------------------
\begin{frame}
\frametitle{Future Research Directions}
\begin{itemize}
  \setlength\itemsep{1.5em}
\item alo polisia
\end{itemize}
\end{frame}
%------------------------------------------------
% REFERENCES?
\end{document}
